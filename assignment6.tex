\documentclass{article}
\usepackage{multirow}
\usepackage{graphicx}
\usepackage{amsmath} % Add this line

\title{Assignment 6: Adjecency list and matrix and breadth-first- and depth-first- search traversal}

\author{Erik Lykke Trier}
\author{Jony Karmakar}
\author{Simen Westegaard Larsen}
\date{\today}

\begin{document}
\maketitle
\subsection*{Exercise 1}
The given adjecency-list representation we can make an adjecency-matrix representation. The start points are indicated
on the left size and the end point is on the top side.

\[
\begin{array}{cc|ccccccc}
    & & \multicolumn{7}{c}{End point} \\
    & & 1 & 2 & 3 & 4 & 5 & 6 & 7 \\
\cline{3-9}
\multirow{7}{*}{\rotatebox{90}{Start point}} & 1 & 0 & 1 & 0 & 1 & 0 & 1 & 0 \\
    & 2 & 0 & 0 & 1 & 0 & 0 & 0 & 0 \\
    & 3 & 1 & 0 & 0 & 0 & 0 & 0 & 0 \\
    & 4 & 0 & 1 & 0 & 0 & 1 & 0 & 0 \\
    & 5 & 0 & 1 & 0 & 0 & 0 & 0 & 0 \\
    & 6 & 0 & 0 & 1 & 0 & 0 & 0 & 1 \\
    & 7 & 0 & 0 & 1 & 0 & 0 & 0 & 0 \\
\end{array}
\]


\subsection*{Exercise 2}
\subsubsection*{2.1}
Traversal of each node using breadth first search traversal. For breadth first traversal we need two queues one to handle which 
nodes we are looking at currently and one for keeping track of the nodes we have already visited. Let us call the queue for the 
current nodes we are looking at "Queue" and the queue keeping track of the already visited "visited". \newline

Step 1: Initially the two queues are empty.   \newline 

Step 2: Push node A to queue and mark it as visited. \newline   

visited: A  \newline

queue: A  \newline \newline

Step 3: Remove A from the queue and visit its unvisited neighbours and add them to the queue.

visited: A, B, C \newline

queue: B, C \newline

Step 4: Remove B from the queue and visit its unvisited neighbours and add them to the queue.

visited: A, B, C, D, E \newline

queue: C, D, E \newline 

Step 5: Remove C from the queue and visit its unvisited neighbours and add them to the queue.

visited: A, B, C, D, E, F \newline

queue: D, E, F \newline

Step 6: Remove D from the queue and visit its unvisited neighbours and add them to the queue.

visited: A, B, C, D, E, F \newline

queue: E, F \newline

Step 7: Remove E from the queue and visit its unvisited neighbours and add them to the queue.

visited: A, B, C, D, E, F \newline

queue: F \newline 
Step 8: Remove F from the queue and visit its unvisited neighbours and add them to the queue.

visited: A, B, C, D, E, F \newline

queue:  \newline

Now that the queue is empty the algorithm will break and all nodes have been visited.

\subsection*{2.2}
Traversal of each node using depth first search traversal. For depth first traversal we need a stack to keep track of the nodes we want to 
visit next by using a stack we will get to the depth before the breadth, and as in BFS we will also keep track of the nodes we have visited
using an array.

Step 1: Initially stack and visited array are empty.
visited: \newline
stack: \newline

Step 2: Push A to visited and its neighbours that we have not yet visited to the stack:
visited: A \newline
stack: B, C \newline

Step 3: Push B to visited and its unvisited neighbours to the stack:
visited: A, B \newline
stack: D, E, C \newline

Step 4: Push D to visited and its unvisited neighbours to the stack:
visited: A, B, D \newline
stack: E, C \newline

Step 5: Push E to visited and its unvisited neighbours to the stack:
visited: A, B, D, E \newline
stack:C \newline

Step 6: Push C to visited and its unvisited neighbours to the stack:
visited: A, B, D, E, C \newline
stack: F \newline

Step 7: Push F to visited and its unvisited neighbours to the stack:
visited: A, B, D, E, C, F \newline
stack: \newline

Stack is now empty so we have visited all the nodes and the DFS traversal is over.




\end{document}